\documentclass[10pt]{beamer}

\usetheme{CambridgeUS}
\usepackage[english, russian]{babel}
\usepackage[utf8]{inputenc}
\usepackage{caption}
\usepackage{etoolbox}
\usepackage{multicol}

\usepackage{listings}
\usepackage{color}

\definecolor{mygreen}{rgb}{0,0.6,0}
\lstset{
  basicstyle=\ttfamily\footnotesize,        % the size of the fonts that are used for the code
  breaklines=true,                 % automatic line breaking only at whitespace
  captionpos=b,                    % sets the caption-position to bottom
  commentstyle=\color{mygreen},    % comment style
  keywordstyle=\color{blue},       % keyword style
  stringstyle=\color{red},     % string literal style
  showstringspaces=false,
  morekeywords={include, printf},
  texcl=true     %<---- added
}

\title[\href{https://goo.gl/NRgp8K}{https://goo.gl/NRgp8K} (Term 3)]{Приведение типов}
\author[Гусев Илья]{Гусев Илья}
\institute[МФТИ] 
{Московский физико-технический институт\\*}
\date{Москва, 2087}
\subject{Computer Science}

\begin{document}

\begin{frame}
  \titlepage
\end{frame}

\begin{frame}{Содержание}
\tableofcontents
\end{frame}

\section{Приведение типов}

\subsection{static\_cast}
\begin{frame}[fragile]{static\_cast}
Можно:
\begin{enumerate}
    \item Преобразование указателя на родительский класс к указателю на дочерний класс. Объект по указателю обязан быть правильного дочернего класса, иначе undefined behaviour. 
    \item Преобразования между числовыми типами. int, long, char, unsigned int — все их можно кастить друг в друга при помощи static\_cast.
    \item Можно закастить любое выражение в void (ради побочных эффектов).
    \item Можно привести константу nullptr к любому типу-указателю.
\end{enumerate}
Нельзя:
\begin{enumerate}
    \item Преобразование между указателями на в принципе несовместимые типы. Например, указатель на double нельзя привести к указателю на int.
    \item Указатели на типы, а также сами типы с несовместимыми атрибутами const и/или volatile.
    \item Привести указатель на функцию-член к указателю на обычную функцию, или указатель на код к указателю на данные.
\end{enumerate}
\end{frame}

\begin{frame}[fragile]{static\_cast - примеры}
\begin{lstlisting}[language=C++]
// Пример числовых преобразований.
double a = 1.0;
int b = static_cast<int>(a);

// Пример каста вверх по иерархии.
class A {};
class B: public A{};
B b;
A* a = static_cast<A*>(&b);

// Пример попытки снять константность.
const int a = 1;
int b = static_cast<int>(a); // Ошибка при компиляции!
\end{lstlisting}
\end{frame}

\subsection{const\_cast}
\begin{frame}[fragile]{const\_cast}
\begin{enumerate}
    \item Добавляет или убирает модификатор const/volatile у переменной.
    \item Если вы изначально определили переменную как const, нельзя на ней использовать const\_cast, иначе UB!
    \item Используется для вызовов правильных перегрузок, например.
    \item Добавить const к неконстантному объекту могут также static\_cast и dynamic\_cast.
\end{enumerate}
\end{frame}

\begin{frame}[fragile]{const\_cast - примеры}
\begin{lstlisting}[language=C++]
int f(const int* a) {
    int* b = const_cast<int*>( a );
    (*b)++;
    return *b;
}

const int c = 1;
int i = 1;

f(c); // UB.
f(i); // Корректно.
\end{lstlisting}
\end{frame}

\subsection{dynamic\_cast}
\begin{frame}[fragile]{dynamic\_cast}
\begin{enumerate}
    \item Нужен для полиморфизма, используется только с классами с виртуальными функциями или виртуальным наследованием.
    \item Если преобразование по иерархии действительно возможно (подходящий изначальный объект), оно произойдёт.
    \item Если преобразование невозможно, и преобразовывались указатели, dynamic\_cast вернёт 0 или nullptr.
    \item Если преобразование невозможно, и перобразовывались ссылки, dynamic\_cast кинет исключение std::bad\_cast.
\end{enumerate}
\end{frame}

\begin{frame}[fragile]{dynamic\_cast - примеры}
\begin{lstlisting}[language=C++]
struct A {
    virtual ~A() {}
};
struct B : A {
    int b = 1;
};

A a;
B b;
A* dynamic_b_a = dynamic_cast<A*>(&b); // OK.
A* static_b_a = static_cast<A*>(&b); // OK.
B* dynamic_a_b = dynamic_cast<B*>(&a); // 0.
B* static_a_b = static_cast<B*>(&a); // UB.
static_a_b->b; // Мусор.
\end{lstlisting}
\end{frame}


\subsection{reinterpret\_cast}
\begin{frame}[fragile]{reinterpret\_cast}
\begin{enumerate}
    \item Самое опасное преобразование, ипользуется для грязных хаков.
    \item Напрямую превращает один тип в другой.
    \item Ограничения: только pointer-to-pointer, reference-to-reference, pointer-to-integer, integer-to-pointer преобразования.
    \item Гарантирует, что если преобразовать сначала в одну сторону, а потом обратно, то получится то же значение. Исключение - если размер типа, в который преобразовываем, меньше исходного.
\end{enumerate}
\end{frame}

\begin{frame}[fragile]{reinterpret\_cast}
\begin{lstlisting}[language=C++]
int a = 1;
// Верхние 4 байта - мусор.
double b = *reinterpret_cast<double*>(&a); // Случайное значение.

if (sizeof(float) == sizeof(int)) {
    int a = 1;
    // Конкретное значение зависит от платформы.
    // Размер типов, порядок байтов.
    float b = *reinterpret_cast<float*>(&a); // 1.4013e-45
}
\end{lstlisting}
\end{frame}

\subsection{С-style cast}
\begin{frame}[fragile]{C-style cast}
Порядок вызова:
\begin{enumerate}
    \item const\_cast.
    \item Если const\_cast не может дать нужный результат, то static\_cast (с игнорированием спецификатора приватного наследования).
    \item Если и так не выходит, то компилятор пробует в хвост к static\_cast добавить const\_cast.
    \item Если и это не получается, то reinterpret\_cast.
    \item Если не выйдет, то к нему дописывается const\_cast.
\end{enumerate}
\end{frame}


\appendix
\section<presentation>*{\appendixname}
\subsection<presentation>*{Useful links}

\begin{frame}[allowframebreaks]
  \frametitle<presentation>{Полезные ссылки}
    
  \begin{thebibliography}{10}
{
  \beamertemplatebookbibitems
  % Start with overview books.
  
  \bibitem{SO1}
  \texttt{StackOverflow: static\_cast}
  \newblock \href{https://goo.gl/GTxYvz}{\texttt{https://ru.stackoverflow.com/questions/387985/Для-чего-\\нужен-static-cast-как-он-работает-и-где-его-применяют}}
  
  \bibitem{SO2}
  \texttt{StackOverflow: When should static\_cast, dynamic\_cast, const\_cast and reinterpret\_cast be used?}
  \newblock \href{https://goo.gl/HHvtjG}{\texttt{https://stackoverflow.com/questions/332030/when-should-static-\\cast-dynamic-cast-const-cast-and-reinterpret-cast-be-used}}
  
}
    
  \end{thebibliography}
\end{frame}

\end{document}


